%░▒█▀▀█░▒█▀▀▄░▒█▀▀▀█░▒█▀▀▄░░░▒█▀▀▀█░▀▀█▀▀░█▀▀▄░▀▀█▀▀░▒█▀▀▀░▒█▀▄▀█░▒█▀▀▀░▒█▄░▒█░▀▀█▀▀
%░▒█▄▄█░▒█▄▄▀░▒█░░▒█░▒█▀▀▄░░░░▀▀▀▄▄░░▒█░░▒█▄▄█░░▒█░░░▒█▀▀▀░▒█▒█▒█░▒█▀▀▀░▒█▒█▒█░░▒█░░
%░▒█░░░░▒█░▒█░▒█▄▄▄█░▒█▄▄█░░░▒█▄▄▄█░░▒█░░▒█░▒█░░▒█░░░▒█▄▄▄░▒█░░▒█░▒█▄▄▄░▒█░░▀█░░▒█░░
%.:..:..:..:..:..:..:..:..:..:..:..:..:..:..:..:..:..:..:..:.
\section{Problem Statement}
\gls{OFDM} became the most popular multiplexing scheme for the following reasons:
\begin{itemize}
	\item Robustness to multi-path fading.
	\item High spectral efficiency.
\end{itemize}
The performance of an \gls{OFDM} system is quantified in its \gls{BER}. This measure is heavily grounded in statistics and probability since analytic estimation would require simultaneous solution of a near-infinite number of sets of Maxwell's equations, and that's before multi-path propagation is considered\cite{wireless_design}.

As a result, computer modelling and physical implementation remain to be the only ways to obtain a \gls{BER} measurement for a specified system. This means that design and testing of \gls{OFDM} systems is only accessible to an elite few who have access to powerful computing hardware, to run simulations or costly lab equipment for the same purpose.

Even having access to the aforementioned equipment, simulation is still quite computationally intensive when the model is accurate in its complexity\cite{wireless_design}. Consequently, even modern powerful processors running highly optimized code take a significant amount of time to complete the simulation. This delay, while bearable for a single pass, is completely unacceptable in an iterative design process.


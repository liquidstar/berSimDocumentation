%░░░░▒█░▒█░▒█░▒█▀▀▀█░▀▀█▀▀░▀█▀░▒█▀▀▀░▀█▀░▒█▀▀▄░█▀▀▄░▀▀█▀▀░▀█▀░▒█▀▀▀█░▒█▄░▒█
%░░░░▒█░▒█░▒█░░▀▀▀▄▄░░▒█░░░▒█░░▒█▀▀░░▒█░░▒█░░░▒█▄▄█░░▒█░░░▒█░░▒█░░▒█░▒█▒█▒█
%░▒█▄▄█░░▀▄▄▀░▒█▄▄▄█░░▒█░░░▄█▄░▒█░░░░▄█▄░▒█▄▄▀▒█░▒█░░▒█░░░▄█▄░▒█▄▄▄█░▒█░░▀█
%.:..:..:..:..:..:..:..:..:..:..:..:..:..:..:..:..:..:..:..:.
\section{Problem Justification}
It has been shown that \gls{OFDM} system analysis is time consuming and costly, even with the right gear. This project aims to come up with an analytic model for the \gls{BER} performance of an \gls{OFDM} system over both \emph{Rayleigh} and \emph{Rician} fading channel models.

The analytic expressions so obtained should allow a communication system designer to establish a rough estimate for the parameters they expect their system to have and from that data, to know the value of the resulting BER for various \gls{SNR} levels. Being an analytic technique, this would permit system analysis on literal paper with relatively inexpensive computational tools such as scientific calculators.

Moreover, where possible, the expression can be evaluated on a computer at relatively negligible cost in processor time, allowing a ludicrously high count of iterations with extraordinary time savings compared to the conventional route.

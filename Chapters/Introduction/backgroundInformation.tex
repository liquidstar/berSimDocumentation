%░▒█▀▀▄░█▀▀▄░▒█▀▀▄░▒█░▄▀░▒█▀▀█░▒█▀▀▄░▒█▀▀▀█░▒█░▒█░▒█▄░▒█░▒█▀▀▄░░░▀█▀░▒█▄░▒█░▒█▀▀▀░▒█▀▀▀█
%░▒█▀▀▄▒█▄▄█░▒█░░░░▒█▀▄░░▒█░▄▄░▒█▄▄▀░▒█░░▒█░▒█░▒█░▒█▒█▒█░▒█░▒█░░░▒█░░▒█▒█▒█░▒█▀▀░░▒█░░▒█
%░▒█▄▄█▒█░▒█░▒█▄▄▀░▒█░▒█░▒█▄▄▀░▒█░▒█░▒█▄▄▄█░░▀▄▄▀░▒█░░▀█░▒█▄▄█░░░▄█▄░▒█░░▀█░▒█░░░░▒█▄▄▄█
%.:..:..:..:..:..:..:..:..:..:..:..:..:..:..:..:..:..:..:..:.
\section{Background Information}
Ceaseless research into existing and prospective communication technologies has resulted in numerous permutations of fundamental standard formats. Thus, for a novel communication system to be considered for standardization and widespread adoption, it has to have an edge over existing implementations. It is for this reason that thorough analyses are performed on communication systems even before they are brought to public light.

In digital communication, information is conveyed in bits which are grouped into symbols. In order to avoid \gls{ISI}, symbol duration must be greater than delay time, by Nyquist's criterion for \gls{ISI} avoidance\cite{ofdm_intro}. Naturally, data rate being inversely proportional to symbol duration implies that longer symbol periods translate to low data rates, and thus diminished communication efficiency. This limitation is especially salient in a \gls{SCM} system. The system designer has to compromise data rate so as to mitigate against \gls{ISI}.

However, in a \gls{MCM} system, total available bandwidth is divided into sub-channels over which multiple sub-carriers can transmit in parallel. Reducing the spectral spacing between the sub-carriers facilitates a higher data rate as more sub-channels can be accommodated. Nevertheless, \gls{ICI} occurs when carrier spacing is too small. In practical \gls{FDM} implementation, a spectral distance between adjacent sub-carriers called a guard band is introduced. While a guard band introduces spectral inefficiencies, it reduces the influence of \gls{ICI}.

\gls{OFDM} is an \gls{MCM} that addresses both the \gls{ISI} problem of high data rate \gls{SCM} schemes and the \gls{ICI} problem of overcrowded-spectrum \gls{MCM} systems. \gls{OFDM} uses a large number of low data rate sub-carriers to accomplish a cumulatively high data rate. The sub-carriers are orthogonally spaced on the spectrum which allows them to overlap without suffering from \gls{ICI}. As each sub-carrier has a relatively low data rate, \gls{ISI} is greatly diminished.

Whereas \gls{OFDM} technology had already been patented in 1966 by \emph{Chang} of \emph{Bell Labs}\cite{ofdm_intro} and earned strong interest, mainstream adoption only just took hold in the 21\textsuperscript{st} century. This is because at the time, implementation was extremely costly.
Thanks to the \gls{FFT} and \gls{IFFT} algorithm pairs alongside an exponential increase in computational power over the years, large scale implementation became viable. This is why in a modern world with an oppressive and stringently regulated spectrum budget, \gls{OFDM} remains to be the preferred multiplexing scheme.

\gls{OFDM} symbols are generated based on the modulation technique used, for example:
\begin{itemize}
	\item M-ary \gls{PSK} modulation
	\item \gls{QAM}
\end{itemize}
Some important \gls{OFDM} system parameters are:
\begin{itemize}
	\item The number of sub-carriers.
	\item Symbol duration, hence sub-carrier spacing.
\end{itemize}
Both parameters are commensurate with the performance and complexity of the \gls{OFDM} system\cite{wireless_design}. In designing an \gls{OFDM} system, coherence bandwidth is to be considered:
\begin{equation}
	B_c > \frac{1}{T_s}
\end{equation}
\begin{mathDef}
	\mathSymb{B_c}{Coherence Bandwidth}
	\mathSymb{T_s}{Symbol Duration}
\end{mathDef}
Under this condition, the OFDM signal only suffers from slow, flat fading.

%██████╗  █████╗ ██████╗ ██████╗ 
%██╔══██╗██╔══██╗██╔══██╗██╔══██╗
%██████╔╝███████║██████╔╝██████╔╝
%██╔═══╝ ██╔══██║██╔═══╝ ██╔══██╗
%██║     ██║  ██║██║     ██║  ██║
%╚═╝     ╚═╝  ╚═╝╚═╝     ╚═╝  ╚═╝
%.:..:..:..:..:..:..:..:..:..:..:..:..:..:..:..:..:..:..:..:.
\section{PAPR Performance}

The \gls{PAPR} was determined according to the definition stated in the literature review: the relation between maximum sample power in an OFDM symbol divided by the symbol's average power. For consistency, individual symbols' PAPRs were averaged to get each variant's equivalent value.

The values obtained are:
\begin{itemize}
	\item \emph{IEEE 802.11 } standard OFDM implementation: \SI{7.25}{\decibel}
	\item \emph{OFDM Variant} \SI{7.35}{\decibel}
\end{itemize}

The increase in average \gls{PAPR} from the standard OFDM implementation to the variant is expected since there are more data sub-carriers in the variant. This means that more signalling energy is expended in transmitting a symbol.
\pagebreak
